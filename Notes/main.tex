\chapter*{Mis notas de estudio}
\section*{Divisibilidad}

\begin{definition}
Un entero $b$ es divisible por un entero $a$, no cero, si existe un entero $x$ tal que $b=ax$ y se escribe a $a\divides b$. En el caso en que $b$ no sea divisible por a se escribe $a\notdivides b$.
\end{definition}

\begin{theorem}
\noindent
Sean $\{a,b,c,x,y\}\subset\mathbb{Z}$, las siguientes proposiciones son verdaderas:
\begin{enumerate}[font={\bfseries},label={1)}]\label{def:1}
	\item Si $a\divides b$, entonces $a\divides bc$ para cualquier entero $c$.
	
	\begin{proof}[Prueba:]
	\noindent
	
	De la definición (\ref{def:1}) se sigue que existe algún entero $m$ tal que $b=a\cdot m$. Ahora, sea $c\in\mathbb{Z}$ fijo y arbitrario. Así, el número $bc=a\cdot m(c)$ y de (\ref{def:1}) existe un entero $d=m(c)$ tal que $b=a\cdot d$, por lo tanto $a\divides bc$.
	\end{proof}

	\item Si $a\divides b$ y $b\divides c$, entonces $a\divides c$.
	
	\begin{proof}[Prueba:]
	\noindent
	
	De la definición (\ref{def:1}) se sigue que existen los entero $m_1$ y $m_2$ tales que $b=a\cdot m_1$ y $c=b\cdot m_2$. Pero $c$ es igual a $b\cdot m_2=(a\cdot m_1)\cdot m_2=a\cdot(m_1\cdot m_2)$, es decir, existe un entero $m_3=m_1\cdot m_2$ tal que $c=a\cdot m_3$, por lo tanto, de (\ref{def:1}) $a\divides c$. 
	\end{proof}

	\item Si $a\divides \left(b_1,b_2,\ldots,b_n\right)$ para algún $n\in\mathbb{N}$, entonces $a\divides \displaystyle\sum_{j=1}^{n}b_jx_j$ para cualesquiera $x_j$.
	
	\begin{proof}[Prueba:]
	\noindent
	
	De la definición (\ref{def:1}) se sigue que existen $n$ números $m_1,m_2,\ldots, m_n$ tales que $b_j=a\cdot m_j$ cuando $j\in\{1,2,\ldots,n\}$.
	\end{proof}

	\item Si $a\divides b$ y $b\divides a$, entonces $a=\pm b$.
	
	\begin{proof}[Prueba:]
	\noindent
	
	\end{proof}
\end{enumerate}
\end{theorem}

\section*{Algunos códigos}

%\inputminted{python}{totient2.py}
\section{Solución de la Ecuación multiplicativa de Cauchy}

La función signo es denotada por $\sgn(x)$ y definida como
\begin{align*}
\sgn(x)=\begin{cases}
1 & \text{si }x>0\\
0 & \text{si }x=0\\
-1 & \text{si }x<0.\\
\end{cases}
\end{align*}

\begin{theorem}
La solución general de la ecuación funcional multiplicativa, esto es,
\[f(xy)=f(x)f(y),\]
sosteniendo para todo $x,y\in\mathbb{R}$ está dada por
\begin{align*}
f(x) = 0,&\\
f(x) = 1,&\\
f(x) = \exp\left(A\left(\ln|x|\right)\right)|\sgn(x)|
\end{align*}
y
\[f(x)=\exp\left(A\left(\ln|x|\right)\right)\sgn(x).\]
donde $A\colon\mathbb{R}\rightarrow\mathbb{R}$ es una función aditiva y $e$ es la base neperiana de los logaritmos.
\begin{proof}
Sea $x=0=y$ en %\eqref{}
,obtenemos $f(0)\left[1-f(0)\right]=0$ y por lo tanto

\[f(0)=0\qquad\text{o}\qquad f(0)=1.\]

Similarmente, sustituyendo $x=1=y$ en %\eqref{}
,obtenemos  $f(1)\left[1-f(1)\right]=0$ y por lo tanto

\[f(1)=0\qquad\text{o}\qquad f(1)=1.\]

Sea $x$ un número real positivo, esto es $x>0$. Entonces %\eqref{}
implica

\[f(x)=f{\left(\sqrt{x}\right)}^{2}\geq0.\]

Suponga que existe un $x_0\in\mathbb{R}, x_0\neq0$ de modo que $f(x_0)=0$. Sea $x\in\mathbb{R}$ un número real arbitrario. Entonces de %\eqref{}
tenemos

\[f(x)=f\left(x_0\frac{x}{x_0}\right)=f(x_0)f\left(\frac{x}{x_0}\right)=0\]

para todo $x\in\mathbb{R}$ y obtenemos la solución %\solref(f(x)=0).

De ahora en adelante, suponemos que $f(x)\neq0$ para todo $x\in\mathbb{R}^{\ast}$.

De %\eqref{}
tenemos cualquiera $f(0)=0$ o $f(0)=1$. Si $f(0)=1$, entonces fijando $y=0$ en %\eqref{2.3}
, obtenemos

\[f(0)=f(x)f(0)\]

y se concluye
\[f(x)=1,\]
para todo $x\in\mathbb{R}$. Finalmente tenemos la solución acertada %\eqref{2.27}
A continuación, consideremos el caso $f(0)=0$. En este caso podemos afirmar que $f$ es en ninguna parte 0. Suponga que no. Entonces, existe un $y_0$ en $\mathbb{R}^{\ast}$ tal que se verifica $f(y_0)=0$. Fijando $y=y_0$ en %\eqref{2.3}
, obtenemos
\[f(xy_0)=f(x)f(y_0)=0.\]
Así
\[f(x)=0\quad\forall x\in\mathbb{R}^{\ast}\]
lo cual es una contradicción a nuestra suposición de que $f$ no es idénticamente cero. Por lo tanto, $f$ está en ninguna parte cero.

Del hecho de que $f$ no está en ninguna parte 0 en $\mathbb{R}^{\ast}$ y %\eqref{2.32}
, obtenemos

\[f(x)>0\quad\text{para}\quad x>0.\]

Sea
\[x=e^{s}\quad\text{y}\quad y=e^{t}\]
así que
\[s=\ln x\quad\text{y}\quad t=\ln y.\]
Note que $s,t\in\mathbb{R}$ dado que $x,y\in\mathbb{R}_{+}$. Sustituyendo % y 
, obtenemos

\[f\left(e^{s+t}\right)=f\left(c^{s}\right)f\left(e^{t}\right).\]

Dado que $f(t)>0$ para todo $t>0$, tomando el logaritmo natural a ambos lados de la ecuación, tenemos
\end{proof}
\end{theorem}
