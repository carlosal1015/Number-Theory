\chapter{Ejercicios}
\section{Lista N$^{\circ}1$}
\begin{enumerate}[font={\bfseries},label={\arabic*.}]
\item Un número racional $a/b$ con $(a,b)=1$ se llama \emph{fracción reducida}. Si la suma de dos fracciones reducidas es un entero, es decir, si $(a/b)+(c/d)=n$. Demostrar que entonces $|b|=|d|$.

\item Si $(a,b)=1$, entonces $(a+b,a-b)$ o es 1 o es 2.

\item Si $(a,b)=1$, entonces $(a+b,a^{2}-ab+b^{2})$ o es 1 o es 3.

\item Si $(a,b)=1$, entonces $(a^{n},b^{k})=1$ para todo $n\geq1, k\geq1$.

\item Un entero se llama \emph{sin cuadrados} si no es divisible por el cuadrado de ningún primo. Probar que, para cada $n\geq1$, existen $a>0$ y $b>0$, unívocamente determinados, tales que $n=a^{2}b$, en donde $b$ es sin cuadrados.

\item Probar que $\dfrac{21n+4}{14n+3}$ es irreducible para todo número natural $n$.

\item Sean $\{a,b,x,y\}\subset\mathbb{N}$. Si $(a,b)=1$ y $ab=c^{n}$, probar que $a=x^{n}$ y $b=y^{n}$ para algunos $x,y$ enteros positivos.

\item Hallar $\left(a^{2^{m}}+1,a^{2^{n}}+1\right)$ en función de $a$.

\item Sean $\{a,b,x,y\}\subset\mathbb{N}$. Si $(a,b)=1$ y $x^{a}=y^{b}$ entonces probar que $x=n^{b}$ e $y=n^{a}$ para algún entero positivo.

\item Si $\{a,m,n\}\subset\mathbb{N}$ con $a>1$, probar que $\left(a^{m}-1,a^{n}-1\right)=a^{(m,n)}-1$.

\item Sea $n$ un entero positivo y sea $S$ un conjunto de enteros positivos menores o iguales a $2n$ tal que si $a$ y $b$ están en $S$ y $a$ y $b$ son diferentes, entonces $a$ no divide a $b$. Hallar el máximo número de elementos de $S$.

\item Hallar todos los pares de enteros positivos $(a,b)$ tales que $a\divides b+1$ y $b\divides a+1$.

\item Hallar todos los pares de enteros positivos $(a,b)$ tales que $a\divides8b+1$ y $b\divides8a+1$.

\item Halle todos los números enteros positivos $n$ tales que el conjunto $\{n,n+1,n+2,n+3,n+4,n+5\}$ puede ser particionado en dos subconjuntos de modo que el producto de los números en cada subconjunto sea igual.

\item Sea $m$ y $n$ números enteros tales que:
\[\frac{m}{n}=1-\frac{1}{2}+\frac{1}{3}-\frac{1}{4}+\cdots-\frac{1}{1318}+\frac{1}{1319}\]

Probar que $m$ es divisible por 1979. Ayuda: 1979 es un número primo.
\end{enumerate}