\documentclass{gemnumber} % Clase del Grupo Estudiantil de Matemática para el curso de Teoría de números.

\newcommand*{\colorboxed}{}
\def\colorboxed#1#{%
	\colorboxedAux{#1}%
}
\newcommand*{\colorboxedAux}[3]{%
	% #1: optional argument for color model
	% #2: color specification
	% #3: formula
	\begingroup
	\colorlet{cb@saved}{.}%
	\color#1{#2}%
	\boxed{%
		\color{cb@saved}%
		#3%
	}%
	\endgroup
}

%\usepackage[citestyle=numeric,style=numeric,backend=biber]{biblatex}
%\addbibresource{num.bib}
\DeclareMathOperator{\mcd}{mcd}
\newcommand{\MVAt}{{\usefont{U}{mvs}{m}{n}\symbol{`@}}} % Se prefiere no usar el paquete marvosym cuando esté cargada mathabx.
\renewcommand{\qedsymbol}{$\blacksquare$}
\newtheorem{definition}{Definición}[chapter] %chapter, section
\newtheorem{theorem}{Teorema}[chapter] %chapter section
\newtheorem{remark}{Observación}[chapter]
\newtheorem{conjecture}{Conjetura}
\theoremstyle{definition}
\pagestyle{empty} % Dentro de este archivo se encuentran los comandos, también puede agregar paquetes según su gusto.

\title{\Huge\bfseries\textcolor{DarkBlue}{Apuntes de clases de Teoría de números}}
\author{\LARGE\textcolor{DarkRed}{Grupo Estudiantil de Matemática}}
\date{\textcolor{DarkMagenta}{Actualizado a la fecha \today}}

\begin{document}

\maketitle

\chapter*{Prefacio}

Estos son los apuntes de clases de Teoría de números organizado por el \href{https://web.facebook.com/GEMFCUNI/}{Grupo Estudiantil de Matemática} durante los meses de enero y febrero del año 2018.

Muchas gracias al \href{http://imca.edu.pe/portal/index.php/es/}{Instituto de Matemática y Ciencias Afines} por brindarnos sus ambientes para llevar a cabo las clases.

Por favor, cualquier sugerencia o aviso de error escribir a {\href{mailto:gem@uni.edu.pe}{gem\MVAt uni.edu.pe}} o {\href{mailto:caznaranl@uni.pe}{caznaranl\MVAt uni.pe}}.

\begin{flushright}
	Carlos Aznarán
	
\end{flushright}

\begin{figure}[h]
	\begin{subfigure}[b]{.5\textwidth}
		\centering
		\captionsetup{justification=centering,margin=0.5cm}
		\includegraphics[height=4\baselineskip,width=.4\linewidth]{signature.png}
		\caption*{\textbf{Und. Jimmy Espinoza Palacios}\\Miembro del GEM\\Facultad de Ciencias}
	\end{subfigure}
	\begin{subfigure}[b]{.5\textwidth}
		\centering
		\captionsetup{justification=centering,margin=0.5cm}
		\includegraphics[height=4\baselineskip,width=.4\linewidth]{signature.png}
		\caption*{\textbf{Und. Bruno Goicochea Vilela}\\Presidente del GEM\\Facultad de Ciencias}
	\end{subfigure}
\end{figure}% Your signature
%El código fuente se puede encontrar en

\newpage

\renewcommand{\contentsname}{Tabla de contenido}

\tableofcontents

\chapter*{Un poco acerca de la historia de la \emph{Teoría de números}}

En el año 1912, durante el quinto 
\href{https://www.mathunion.org/organization/imu-history}{Congreso Internacional de Matemáticos}, el matemático alemán, en tal evento, \href{https://en.wikipedia.org/wiki/Edmund_Landau}{Edmund Landau} listó cuatro problemas básicos acerca de \emph{los números primos} que los calificó como \emph{inabarcable en el estado actual de las matemáticas} y en nuestros días es conocido como los \textbf{\textit{Problemas de Landau}}. Ellos son los siguientes:
\begin{enumerate}
	\item Conjetura de Goldbach\footnote{Helfgott probó la conjetura débil en el año 2013.}
	\item Conjetura de los primos gemelos.
	\item Conjetura de Legendre.
	\item ¿Existen infinitos números primos de la forma $n^{2}+1$?
\end{enumerate}

Ninguno de los cuatro problemas han sido resultados a la fecha de la edición de los apuntes.

La teoría de números tal como se conoce en la actualidad inició desde Gauss. En el año 2013, el matemático peruano Harald Andrés Helfgott demostró la \emph{Conjetura débil (o ternaria) de Goldbach}
%https://www.gaussianos.com/harald-andres-helfgott-nos-habla-sobre-su-demostracion-de-la-conjetura-debil-de-goldbach/
\section{Harald Andrés Helfgott}
\begin{conjecture}{Goldbach}
Todo número impar mayor que cinco es la suma de tres números primos.
\end{conjecture}
\emph{Leonhard Euler}, uno de los más grandes matemáticos del siglo XVIII, y de todas las épocas, y su amigo cercano \emph{Christian Goldbach}, quien tenía grandes conocimientos tanto en la ciencia como en las humanidades, mantuvo una regular y copiosa correspondencia. Goldbach hizo una conjetura acerca de los números primos, y Euler rápidamente redujo a la siguiente conjetura, el cual, dijo, Goldbach ya le había dicho: cada entero positivo puede escribirse a lo más, como la suma de tres números primos.

%En el año X, Terence Tao probó para el caso como la suma de cinco números primos.

Ahora, podríamos decir que ``cada entero mayor que cinco'', ya que no consideramos a 1 como un número primo. Es más, en la actualidad la conjetura se divide en dos casos: la \emph{conjetura débil o ternaria de Goldbach} establece que cada entero impar mayor que cinco se puede escribir como la suma de tres primos y la \emph{conjetura fuerte o binaria de Goldbach} establece que cada entero par mayor que 2 se puede escribir como la suma de dos primos. Como indican sus nombres, la conjetura fuerte implica la débil (fácilmente: reste 3 al número impar $n$, luego exprese $n-3$ como la suma de dos números primos).
%https://valuevar.wordpress.com/2013/07/02/the-ternary-goldbach-conjecture/
%http://libgen.io/search.php?&req=History+of+the+theory+of+numbers+Dickson&phrase=1&view=simple&column=def&sort=year&sortmode=DESC
%Cayley y Sylvester trabajaron en la conjetura de Goldbach.
%Citar el libro de Escalante sobre las figuras.
%Grandes matemáticos de Temple Bell.

\begin{figure}[H]
	\centering
	\def\svgwidth{7cm}
	\input{imo2018.pdf_tex}	
\end{figure}



\input{first.tex}

\chapter{Ejercicios}
\section{Lista N$^{\circ}1$}
\begin{enumerate}[font={\bfseries},label={\arabic*.}]
\item Un número racional $a/b$ con $(a,b)=1$ se llama \emph{fracción reducida}. Si la suma de dos fracciones reducidas es un entero, es decir, si $(a/b)+(c/d)=n$. Demostrar que entonces $|b|=|d|$.

\item Si $(a,b)=1$, entonces $(a+b,a-b)$ o es 1 o es 2.

\item Si $(a,b)=1$, entonces $(a+b,a^{2}-ab+b^{2})$ o es 1 o es 3.

\item Si $(a,b)=1$, entonces $(a^{n},b^{k})=1$ para todo $n\geq1, k\geq1$.

\item Un entero se llama \emph{sin cuadrados} si no es divisible por el cuadrado de ningún primo. Probar que, para cada $n\geq1$, existen $a>0$ y $b>0$, unívocamente determinados, tales que $n=a^{2}b$, en donde $b$ es sin cuadrados.

\item Probar que $\dfrac{21n+4}{14n+3}$ es irreducible para todo número natural $n$.

\item Sean $\{a,b,x,y\}\subset\mathbb{N}$. Si $(a,b)=1$ y $ab=c^{n}$, probar que $a=x^{n}$ y $b=y^{n}$ para algunos $x,y$ enteros positivos.

\item Hallar $\left(a^{2^{m}}+1,a^{2^{n}}+1\right)$ en función de $a$.

\item Sean $\{a,b,x,y\}\subset\mathbb{N}$. Si $(a,b)=1$ y $x^{a}=y^{b}$ entonces probar que $x=n^{b}$ e $y=n^{a}$ para algún entero positivo.

\item Si $\{a,m,n\}\subset\mathbb{N}$ con $a>1$, probar que $\left(a^{m}-1,a^{n}-1\right)=a^{(m,n)}-1$.

\item Sea $n$ un entero positivo y sea $S$ un conjunto de enteros positivos menores o iguales a $2n$ tal que si $a$ y $b$ están en $S$ y $a$ y $b$ son diferentes, entonces $a$ no divide a $b$. Hallar el máximo número de elementos de $S$.

\item Hallar todos los pares de enteros positivos $(a,b)$ tales que $a\divides b+1$ y $b\divides a+1$.

\item Hallar todos los pares de enteros positivos $(a,b)$ tales que $a\divides8b+1$ y $b\divides8a+1$.

\item Halle todos los números enteros positivos $n$ tales que el conjunto $\{n,n+1,n+2,n+3,n+4,n+5\}$ puede ser particionado en dos subconjuntos de modo que el producto de los números en cada subconjunto sea igual.

\item Sea $m$ y $n$ números enteros tales que:
\[\frac{m}{n}=1-\frac{1}{2}+\frac{1}{3}-\frac{1}{4}+\cdots-\frac{1}{1318}+\frac{1}{1319}\]

Probar que $m$ es divisible por 1979. Ayuda: 1979 es un número primo.
\end{enumerate}

\chapter{Funciones aritméticas}

\begin{definition}
Una \emph{función aritmética} es cualquiera función $f\colon\mathbb{Z}^{+}\rightarrow\mathbb{C}$.	
\end{definition}

Algunas funciones aritméticas son:

\begin{enumerate}
	\item Función de M\"{o}bius $\mu(n)$
	\[\mu\colon\mathbb{Z}^{+}\longrightarrow\{-1,0,1\}\]
	\[
	\mu(n)= 
	\begin{cases}
	1 & \text{si } n=1,\\
	{(-1)}^{k}  & \text{si } \dc\text{ con } \alpha_1=\alpha_2=\cdots=\alpha_k=1;\\
	0&\text{en otro caso.}
	\end{cases}
	\]
\end{enumerate}

\begin{remark} $\mu(n)=0$ si y solo si $n$ posee un divisor cuadrado perfecto mayor que 1.
\end{remark}
\begin{example}
	Algunos valores de la función $\varphi(n)\colon$
	
\begin{table}[H]
	\centering
	\begin{tabular}{ccccccccccc}
		$n\colon$ &1 &2 &3 & 4 & 5 & 6 & 7 & 8 &9 & 10\\[0.1cm]
		$\mu(n)\colon$ &1 &-1 & -1 & 0 & -1 & 1 & -1 & 0 & 0 & 1
	\end{tabular}
\end{table}

\end{example}

\begin{theorem}
	Si $n\geq1$ y $d>0$, entonces $\displaystyle\sum_{d\divides n}\mu(d)= 
	\begin{cases}
	1 & \text{si } n=1,\\
	0 &\text{si } n>1.
	\end{cases}$

\begin{proof}[Demostración]
	\begin{enumerate}
		\item Si $n=1\colon$ $\underbrace{\mu(1)}_{\displaystyle\sum_{d\divides 1}\mu(d)}=1$ \checkmark
		
		\item Si $n>1$, entonces $\dc$. Si $d\divides n$ y $d$ posee un factor cuadrado perfecto mayor que 1, entonces $\mu(d)=0$.
		\[
		\sum_{d\divides n}\mu(d)=\underbrace{\cancelto{0}{\sum_{d\divides n}\mu(d)}}_{\displaystyle d\text{ posee algún factor }k^{2}>1}+\underbrace{\sum_{d\divides n}\mu(d)}_{\displaystyle d\text{ no posee factor }k^{2}>1}
		\]
		$d\divides n$ y $d$ no poseen factor $k^{2}>1$.
		
		$\displaystyle\implies d=p_{i1}p_{i2}\cdots p_{is}$
		\begin{align*}
		\implies \sum_{d\divides n}\mu(d)
		&=\mu(1)+\mu(p_1)+\mu(p_2)+\cdots+\mu(p_k)+&\\
		& \mu(p_1p_2)+\cdots+\mu(p_{k-1}p_{k})+\cdots\mu(p_1p_2\cdots p_k)&\\
		&=1\binom{k}{0}+(-1)\binom{k}{1}+{(-1)}^{2}\binom{k}{2}+\cdots+{(-1)}^{k}\binom{k}{k}&\\
		&={(1+(-1))}^{k}=0.&
		\end{align*}
	\end{enumerate}
\end{proof}
\end{theorem}

\begin{enumerate}
	\item Función indicador de Euler.
	\[\varphi\colon\mathbb{Z}^{+}\longrightarrow\mathbb{Z}^{+}\]
\end{enumerate}

$\varphi(n)\colon$ cantidad de números menores o iguales que $n$ y coprimos con $n$.

\begin{example}
	
\begin{table}[H]
	\centering
	\begin{tabular}{ccccccccccc}
		$n\colon$ &1 &2 &3 & 4 & 5 & 6 & 7 & 8 &9 & 10\\[0.1cm]
		$\varphi(n)\colon$ &1 &1 & 2 &2 & 4 & 2 & 6 & 4 & 6 & 4
	\end{tabular}
\end{table}	

\end{example}

\begin{theorem}
	Si $n\geq1$, entonces $\displaystyle\sum_{d\divides n}\varphi(d)=n$.
\begin{proof}[Demostración]
	Para cada $d$ tal que $d\divides n$.
	Sea: $\mathcal{A}_d\coloneq\{k\colon\mcd(k,n)=d, 1\leq k\leq n\}$.
	
	$a\in\mathcal{A}_d, \mathcal{A}_{d^{\prime}} (d\neq d^{\prime})$
	
	$\implies \mcd(a,n)=d, \mcd(a,n)=d^{\prime} (\implies\impliedby)$.
	
	$\implies \mathcal{A}_{d}\bigcap\mathcal{A}_{d^{\prime}}=\emptyset (\forall d\neq d^{\prime})$
	
	Sea $f(d)$ la cantidad de elementos de $\mathcal{A}_{d}$.
	%inclusion derecha
	\[\bigcup_{d\divides n}\mathcal{A}_{d}=\{1,2,\ldots,n\}.%={\{i\}}_{i=1}^{n}
	\]
	Porque si $k\leq n\implies \mcd(k,n)=d$ en donde $d\divides n\implies \mathcal{A}_{d}\subset\bigcup_{d\divides n}\mathcal{A}_{d}\implies\displaystyle\sum_{d\divides n}f(d)=n$.
	%Inclusión izquierda
	Pero $\mcd(k,n)=d\implies\mcd\left(\frac{k}{d},\frac{n}{d}\right)=1$.
	
	\[\mathcal{A}_{d}\coloneq\left\{k\colon\left(\frac{k}{d},\frac{n}{d}\right)=1, 1\leq k\leq n\right\}\]
	
	\[\mathcal{B}_{d}\coloneq\left\{q\colon\left(q,\frac{n}{d}\right)=1, 1\leq q\leq \frac{n}{d}\right\}\]
	
	$\implies|\mathcal{A}_{d}|=|\mathcal{B}_{d}|=f(d)=\varphi\left(\frac{n}{d}\right)$
	
	$\displaystyle\implies\sum_{d\divides n}\varphi\left(\frac{n}{d}\right)=n$.
\end{proof}
\end{theorem}

\section{Relación entre $\mu$ y $\varphi$}

\begin{theorem}
	Si $n\geq1$, entonces $\displaystyle\varphi(n)=\sum_{d\divides n}\mu(d)\left(\frac{n}{d}\right)$.
	\begin{proof}[Demostración]
		\[\varphi(n)=\sum_{k=1}^{n}i(k), \text{ donde } i(k)=\begin{cases}
		1 & \text{si } \mcd(n,k)=1,\\
		0 & \text{si } \mcd(n,k)\neq1.
		\end{cases}\]
		Por el teorema:
		\[\sum_{d\divides\mcd(n,k)}\mu(d)=\begin{cases}
		1 & \text{si }\mcd(n,k)=1,\\
		0 & \text{si } \mcd(n,k)>1.
		\end{cases}\]
		
		\[\sum_{d\divides\mcd(n,k)}\mu(d)=\begin{cases}
		1 & \text{si } i(k)=1,\\
		0 & \text{si } i(k)=0.
		\end{cases}\]
		
		\[\sum_{d\divides\mcd(n,k)}\mu(d)=i(k)\]
		
		\[\implies\varphi(n)=\sum_{d\divides\mcd(n,k)}\mu(d)=\sum_{k=1}^{n}\sum_{\substack{d\divides n\\
		d\divides k}}\mu(d)\]
	
		Fijamos un divisor $d$ en $n$. Entonces, el divisor $d$ de $n$ aparecerá siempre y cuando $k$ sea múltiplo de $d$ $(k=qd)$. Por lo tanto,
		\[d\leq k\leq n\implies 1\leq q\leq\frac{n}{d}.\]
		Hay $\dfrac{n}{d}$ múltiplos de $k$.
		\[\implies\varphi(n)=\sum_{d\divides n}\mu(d)\frac{n}{d}.\]
	\end{proof}
\end{theorem}

\subsection{Fórmula para $\varphi(n)$}

\begin{theorem}
Si $n>1$, entonces $\displaystyle\varphi(n)=n\prod\left(1-\frac{1}{p}\right)$.
\begin{proof}[Demostración]
Usar el principio de inclusión y exclusión. Sean $\mathcal{A}_1,\mathcal{A}_2,\mathcal{A}_3,\ldots,\mathcal{A}_k$ conjuntos (puede haber algún conjunto vacío).

\[\displaystyle\implies\left|\bigcup^{k}_{i=1}\mathcal{A}_{i}\right|=\sum_{i=1}^{k}\left|\mathcal{A}_i\right|-\sum_{\substack{i,j=1\\i\neq j}}^{k}\left|\mathcal{A}_{i}\bigcap\mathcal{A}_{j}\right|+\sum_{i,j,\ell=1}\left|\mathcal{A}_i\bigcup\right| \]
	\end{proof}
\end{theorem}

\chapter*{Mis notas de estudio}
\section*{Divisibilidad}

\begin{definition}
Un entero $b$ es divisible por un entero $a$, no cero, si existe un entero $x$ tal que $b=ax$ y se escribe a $a\divides b$. En el caso en que $b$ no sea divisible por a se escribe $a\notdivides b$.
\end{definition}

\begin{theorem}
\noindent
Sean $\{a,b,c,x,y\}\subset\mathbb{Z}$, las siguientes proposiciones son verdaderas:
\begin{enumerate}[font={\bfseries},label={1)}]\label{def:1}
	\item Si $a\divides b$, entonces $a\divides bc$ para cualquier entero $c$.
	
	\begin{proof}[Prueba:]
	\noindent
	
	De la definición (\ref{def:1}) se sigue que existe algún entero $m$ tal que $b=a\cdot m$. Ahora, sea $c\in\mathbb{Z}$ fijo y arbitrario. Así, el número $bc=a\cdot m(c)$ y de (\ref{def:1}) existe un entero $d=m(c)$ tal que $b=a\cdot d$, por lo tanto $a\divides bc$.
	\end{proof}

	\item Si $a\divides b$ y $b\divides c$, entonces $a\divides c$.
	
	\begin{proof}[Prueba:]
	\noindent
	
	De la definición (\ref{def:1}) se sigue que existen los entero $m_1$ y $m_2$ tales que $b=a\cdot m_1$ y $c=b\cdot m_2$. Pero $c$ es igual a $b\cdot m_2=(a\cdot m_1)\cdot m_2=a\cdot(m_1\cdot m_2)$, es decir, existe un entero $m_3=m_1\cdot m_2$ tal que $c=a\cdot m_3$, por lo tanto, de (\ref{def:1}) $a\divides c$. 
	\end{proof}

	\item Si $a\divides \left(b_1,b_2,\ldots,b_n\right)$ para algún $n\in\mathbb{N}$, entonces $a\divides \displaystyle\sum_{j=1}^{n}b_jx_j$ para cualesquiera $x_j$.
	
	\begin{proof}[Prueba:]
	\noindent
	
	De la definición (\ref{def:1}) se sigue que existen $n$ números $m_1,m_2,\ldots, m_n$ tales que $b_j=a\cdot m_j$ cuando $j\in\{1,2,\ldots,n\}$.
	\end{proof}

	\item Si $a\divides b$ y $b\divides a$, entonces $a=\pm b$.
	
	\begin{proof}[Prueba:]
	\noindent
	
	\end{proof}
\end{enumerate}
\end{theorem}

\section*{Algunos códigos}

%\inputminted{python}{totient2.py}
\section{Solución de la Ecuación multiplicativa de Cauchy}

La función signo es denotada por $\sgn(x)$ y definida como
\begin{align*}
\sgn(x)=\begin{cases}
1 & \text{si }x>0\\
0 & \text{si }x=0\\
-1 & \text{si }x<0.\\
\end{cases}
\end{align*}

\begin{theorem}
La solución general de la ecuación funcional multiplicativa, esto es,
\[f(xy)=f(x)f(y),\]
sosteniendo para todo $x,y\in\mathbb{R}$ está dada por
\begin{align*}
f(x) = 0,&\\
f(x) = 1,&\\
f(x) = \exp\left(A\left(\ln|x|\right)\right)|\sgn(x)|
\end{align*}
y
\[f(x)=\exp\left(A\left(\ln|x|\right)\right)\sgn(x).\]
donde $A\colon\mathbb{R}\rightarrow\mathbb{R}$ es una función aditiva y $e$ es la base neperiana de los logaritmos.
\begin{proof}
Sea $x=0=y$ en %\eqref{}
,obtenemos $f(0)\left[1-f(0)\right]=0$ y por lo tanto

\[f(0)=0\qquad\text{o}\qquad f(0)=1.\]

Similarmente, sustituyendo $x=1=y$ en %\eqref{}
,obtenemos  $f(1)\left[1-f(1)\right]=0$ y por lo tanto

\[f(1)=0\qquad\text{o}\qquad f(1)=1.\]

Sea $x$ un número real positivo, esto es $x>0$. Entonces %\eqref{}
implica

\[f(x)=f{\left(\sqrt{x}\right)}^{2}\geq0.\]

Suponga que existe un $x_0\in\mathbb{R}, x_0\neq0$ de modo que $f(x_0)=0$. Sea $x\in\mathbb{R}$ un número real arbitrario. Entonces de %\eqref{}
tenemos

\[f(x)=f\left(x_0\frac{x}{x_0}\right)=f(x_0)f\left(\frac{x}{x_0}\right)=0\]

para todo $x\in\mathbb{R}$ y obtenemos la solución %\solref(f(x)=0).

De ahora en adelante, suponemos que $f(x)\neq0$ para todo $x\in\mathbb{R}^{\ast}$.

De %\eqref{}
tenemos cualquiera $f(0)=0$ o $f(0)=1$. Si $f(0)=1$, entonces fijando $y=0$ en %\eqref{2.3}
, obtenemos

\[f(0)=f(x)f(0)\]

y se concluye
\[f(x)=1,\]
para todo $x\in\mathbb{R}$. Finalmente tenemos la solución acertada %\eqref{2.27}
A continuación, consideremos el caso $f(0)=0$. En este caso podemos afirmar que $f$ es en ninguna parte 0. Suponga que no. Entonces, existe un $y_0$ en $\mathbb{R}^{\ast}$ tal que se verifica $f(y_0)=0$. Fijando $y=y_0$ en %\eqref{2.3}
, obtenemos
\[f(xy_0)=f(x)f(y_0)=0.\]
Así
\[f(x)=0\quad\forall x\in\mathbb{R}^{\ast}\]
lo cual es una contradicción a nuestra suposición de que $f$ no es idénticamente cero. Por lo tanto, $f$ está en ninguna parte cero.

Del hecho de que $f$ no está en ninguna parte 0 en $\mathbb{R}^{\ast}$ y %\eqref{2.32}
, obtenemos

\[f(x)>0\quad\text{para}\quad x>0.\]

Sea
\[x=e^{s}\quad\text{y}\quad y=e^{t}\]
así que
\[s=\ln x\quad\text{y}\quad t=\ln y.\]
Note que $s,t\in\mathbb{R}$ dado que $x,y\in\mathbb{R}_{+}$. Sustituyendo % y 
, obtenemos

\[f\left(e^{s+t}\right)=f\left(c^{s}\right)f\left(e^{t}\right).\]

Dado que $f(t)>0$ para todo $t>0$, tomando el logaritmo natural a ambos lados de la ecuación, tenemos
\end{proof}
\end{theorem}


Un divisor positivo de $n$, el cual ni es 1 ni $n$, es llamado un \textbf{divisor propio}, y un \textbf{primo} es un entero mayor que 1 que no tiene divisores propios.

\end{document}