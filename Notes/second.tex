\chapter{Funciones aritméticas}

\begin{definition}
Una \textbf{función aritmética} es cualquiera función $f\colon\mathbb{Z}^{+}\rightarrow\mathbb{C}$.	
\end{definition}

Algunas funciones aritméticas son:

\begin{enumerate}
	\item Función de M\"{o}bius $\mu(n)$
	\[\mu\colon\mathbb{Z}^{+}\longrightarrow\{-1,0,1\}\]
	\[\mu(n)= 
	\begin{cases}
	1 & \text{si } n=1,\\
	{(-1)}^{k}  & \text{si } \dc\text{ con } \alpha_1=\alpha_2=\cdots=\alpha_k=1;\\
	0&\text{en otro caso.}
	\end{cases}\]
\begin{remark} $\mu(n)=0$ si y solo si $n$ posee un divisor cuadrado perfecto mayor que 1.
\end{remark}
\begin{example}
	Algunos valores de la función $\mu(n)\colon$
	
\begin{table}[H]
	\centering
	\begin{tabular}{ccccccccccc}
		$n\colon$ &1 &2 &3 & 4 & 5 & 6 & 7 & 8 &9 & 10\\[0.1cm]
		$\mu(n)\colon$ &1 &-1 & -1 & 0 & -1 & 1 & -1 & 0 & 0 & 1
	\end{tabular}
\end{table}

\end{example}

\begin{theorem}
	Si $n\geq1$ y $d>0$, entonces $\displaystyle\sum_{d\divides n}\mu(d)= 
	\begin{cases}
	1 & \text{si } n=1,\\
	0 &\text{si } n>1.
	\end{cases}$

\begin{proof}[Demostración]

	\begin{enumerate}
		\item Si $n=1\colon$ $\underbrace{\mu(1)}_{\displaystyle\sum_{d\divides 1}\mu(d)}=1$ \checkmark
		
		\item Si $n>1$, entonces $\dc$. Si $d\divides n$ y $d$ posee un factor cuadrado perfecto mayor que 1, entonces $\mu(d)=0$.
		\[
		\sum_{d\divides n}\mu(d)=\underbrace{\cancelto{0}{\sum_{d\divides n}\mu(d)}}_{\displaystyle d\text{ posee algún factor }k^{2}>1}+\underbrace{\sum_{d\divides n}\mu(d)}_{\displaystyle d\text{ no posee factor }k^{2}>1}
		\]
		$d\divides n$ y $d$ no poseen factor $k^{2}>1$.
		
		$\displaystyle\implies d=p_{i1}p_{i2}\cdots p_{is}$
		\begin{align*}
		\implies \sum_{d\divides n}\mu(d)
		&=\mu(1)+\mu(p_1)+\mu(p_2)+\cdots+\mu(p_k)+&\\
		& \mu(p_1p_2)+\cdots+\mu(p_{k-1}p_{k})+\cdots\mu(p_1p_2\cdots p_k)&\\
		&=1\binom{k}{0}+(-1)\binom{k}{1}+{(-1)}^{2}\binom{k}{2}+\cdots+{(-1)}^{k}\binom{k}{k}&\\
		&={(1+(-1))}^{k}=0.&
		\end{align*}
	\end{enumerate}
\end{proof}
\end{theorem}

	\item Función indicador de Euler.
	\[\varphi\colon\mathbb{Z}^{+}\longrightarrow\mathbb{Z}^{+}\]
\end{enumerate}

$\varphi(n)\colon$ cantidad de números menores o iguales que $n$ y coprimos con $n$.

\begin{example}
	Algunos valores de la función $\varphi(n)\colon$
\begin{table}[H]
	\centering
	\begin{tabular}{ccccccccccc}
		$n\colon$ &1 &2 &3 & 4 & 5 & 6 & 7 & 8 &9 & 10\\[0.1cm]
		$\varphi(n)\colon$ &1 &1 & 2 &2 & 4 & 2 & 6 & 4 & 6 & 4
	\end{tabular}
\end{table}	

\end{example}

\begin{theorem}
	Si $n\geq1$, entonces $\displaystyle\sum_{d\divides n}\varphi(d)=n$.
\begin{proof}[Demostración]
	Para cada $d$ tal que $d\divides n$. Sea$\colon$
	
	\[\mathcal{A}_d\coloneq\{k\colon\mcd(k,n)=d, 1\leq k\leq n\}.\]
	
	$a\in\mathcal{A}_d, \mathcal{A}_{d^{\prime}} (d\neq d^{\prime})$
	
	$\implies \mcd(a,n)=d, \mcd(a,n)=d^{\prime} (\implies\impliedby)$.
	
	$\implies \mathcal{A}_{d}\bigcap\mathcal{A}_{d^{\prime}}=\emptyset (\forall d\neq d^{\prime})$
	
	Sea $f(d)$ la cantidad de elementos de $\mathcal{A}_{d}$.
	%inclusion derecha
	\[\bigcup_{d\divides n}\mathcal{A}_{d}=\{1,2,\ldots,n\}.%={\{i\}}_{i=1}^{n}
	\]
	Porque si $k\leq n\implies \mcd(k,n)=d$ en donde $d\divides n\implies \mathcal{A}_{d}\subset\bigcup_{d\divides n}\mathcal{A}_{d}\implies\displaystyle\sum_{d\divides n}f(d)=n$.
	%Inclusión izquierda
	Pero $\mcd(k,n)=d\implies\mcd\left(\frac{k}{d},\frac{n}{d}\right)=1$.
	
	\[\mathcal{A}_{d}\coloneq\left\{k\colon\left(\frac{k}{d},\frac{n}{d}\right)=1, 1\leq k\leq n\right\}\]
	
	\[\mathcal{B}_{d}\coloneq\left\{q\colon\left(q,\frac{n}{d}\right)=1, 1\leq q\leq \frac{n}{d}\right\}\]
	
	$\implies|\mathcal{A}_{d}|=|\mathcal{B}_{d}|=f(d)=\varphi\left(\frac{n}{d}\right)$
	
	$\boxed{\displaystyle\implies\sum_{d\divides n}\varphi\left(\frac{n}{d}\right)=n.}$
\end{proof}
\end{theorem}

\section{Relación entre $\mu$ y $\varphi$}

\begin{theorem}
	Si $n\geq1$, entonces $\displaystyle\varphi(n)=\sum_{d\divides n}\mu(d)\left(\frac{n}{d}\right)$.
	\begin{proof}[Demostración]
		\[\varphi(n)=\sum_{k=1}^{n}i(k), \text{ donde } i(k)=\begin{cases}
		1 & \text{si } \mcd(n,k)=1,\\
		0 & \text{si } \mcd(n,k)\neq1.
		\end{cases}\]
		Por el teorema:
		\[\sum_{d\divides\mcd(n,k)}\mu(d)=\begin{cases}
		1 & \text{si }\mcd(n,k)=1,\\
		0 & \text{si } \mcd(n,k)>1.
		\end{cases}\]
		
		\[\sum_{d\divides\mcd(n,k)}\mu(d)=\begin{cases}
		1 & \text{si } i(k)=1,\\
		0 & \text{si } i(k)=0.
		\end{cases}\]
		
		\[\sum_{d\divides\mcd(n,k)}\mu(d)=i(k)\]
		
		\[\implies\varphi(n)=\sum_{d\divides\mcd(n,k)}\mu(d)=\sum_{k=1}^{n}\sum_{\substack{d\divides n\\
		d\divides k}}\mu(d)\]
	
		Fijamos un divisor $d$ en $n$. Entonces, el divisor $d$ de $n$ aparecerá siempre y cuando $k$ sea múltiplo de $d$ $(k=qd)$. Por lo tanto,
		\[d\leq k\leq n\implies 1\leq q\leq\frac{n}{d}.\]
		Hay $\dfrac{n}{d}$ múltiplos de $k$.
		\[\implies\varphi(n)=\sum_{d\divides n}\mu(d)\frac{n}{d}.\]
	\end{proof}
\end{theorem}

\subsection{Fórmula para $\varphi(n)$}

\begin{theorem}
Si $n>1$, entonces $\displaystyle\varphi(n)=n\prod\left(1-\frac{1}{p}\right)$.
\begin{proof}[Demostración]
Usar el principio de inclusión y exclusión. Sean $\mathcal{A}_1,\mathcal{A}_2,\mathcal{A}_3,\ldots,\mathcal{A}_k$ conjuntos (puede haber algún conjunto vacío).

\[\displaystyle\implies\left|\bigcup^{k}_{i=1}\mathcal{A}_{i}\right|=\sum_{i=1}^{k}\left|\mathcal{A}_i\right|-\sum_{\substack{i,j=1\\i\neq j}}^{k}\left|\mathcal{A}_{i}\bigcap\mathcal{A}_{j}\right|+\sum_{i,j,\ell=1}\left|\mathcal{A}_i\bigcup\right| \]

\[\left|\mathcal{A}_1\bigcup\mathcal{A}_2\bigcup\cdots\bigcup\mathcal{A}_k\right|=\sum_{i=1}^{k}{(-1)}^{i+1}{\left|\bigcap_{k=1}^{i}\mathcal{A}_{\tilde{t}}\right|}_{1\leq t_1\leq t_2\leq\cdots\leq t_i\leq k}\]

$\dc$

\begin{align*}
\mathcal{A}_1 &= \{\text{ múltiplos }p_1\leq n \}, &|A_1|&=\frac{n}{p_1}&\\
\mathcal{A}_2 &= \{\text{ múltiplos }p_2\leq n \}, &|A_2|&=\frac{n}{p_2}&\\
\mathcal{A}_3 &= \{\text{ múltiplos }p_3\leq n \}, &|A_3|&=\frac{n}{p_3}&\\
\vdots &= \vdots,&|A_1|&=\frac{n}{p_1}&\\
\mathcal{A}_k &=\{ \text{ múltiplos }p_k\leq n \}, &|A_k|&=\frac{n}{p_k}&\\
\end{align*}

El conjunto $\mathcal{A}_1\bigcup\mathcal{A}_2\bigcup\mathcal{A}_3\bigcup\cdots\bigcup\mathcal{A}_k$ tiene algún factor común con $n$. Por lo tanto,  ${\left(\mathcal{A}_1\bigcup\mathcal{A}_2\bigcup\mathcal{A}_3\bigcup\cdots\bigcup\mathcal{A}_k\right)}^{\prime}\colon$elementos coprimos con $n$ y $\leq n$.

$\implies \varphi(n)=\left|{\left(\mathcal{A}_1\bigcup\mathcal{A}_2\bigcup\mathcal{A}_3\bigcup\cdots\bigcup\mathcal{A}_k\right)}^{\prime}\right|=n-\left|\mathcal{A}_1\bigcup\mathcal{A}_2\bigcup\mathcal{A}_3\bigcup\cdots\bigcup\mathcal{A}_k\right|$

Pero:

\[\left|\mathcal{A}_1\bigcup\mathcal{A}_2\bigcup\mathcal{A}_3\bigcup\cdots\bigcup\mathcal{A}_k\right|=|\mathcal{A}_1|+|\mathcal{A}_2|+\cdots+|\mathcal{A}_k|-\left(\left|\mathcal{A}_1\bigcup\mathcal{A}_2\right|+\cdots+\left|\mathcal{A}_{k-1}\bigcup\mathcal{A}_k\right|\right)+\cdots+{(-1)}^{k+1}\left|\mathcal{A}_1\bigcap\mathcal{A}_2\bigcap\cdots\bigcap\mathcal{A}_k\right|
\]

El conjunto:
\[\mathcal{A}_{i_1}\bigcap\mathcal{A}_{i_2}\bigcap\cdots\bigcap\mathcal{A}_{i_t}=\frac{n}{p_{i_1}p_{i_2}\cdots p_{i_t}}\]

\[\left|\mathcal{A}_1\bigcap\mathcal{A}_2\bigcap\cdots\bigcap\mathcal{A}_k\right|=\frac{n}{p_1}+\frac{n}{p_2}+\cdots+\frac{n}{p_k}-\left(\frac{n}{p_1p_2}+\cdots+\frac{n}{p_{k-1}p_{k}}\right)+\cdots+{(-1)}^{k-1}\left(\frac{n}{p_1p_2\cdots p_k}\right)%-1 a la k+1 también es equivalente.
\]
\begin{align*}
\implies \varphi(n)
&=n-\left|\mathcal{A}_1\bigcup\mathcal{A}_2\bigcup\cdots\bigcup\mathcal{A}_k\right|&\\
&=n-\left(\frac{n}{p_1}+\frac{n}{p_2}+\cdots+\frac{n}{p_k}\right)+\left(\frac{n}{p_1p_2}+\cdots+\frac{n}{p_{k-1}p_k}\right)-+\cdots+{(-1)}^{k-1}\cdot\frac{n}{p_1p_2\cdots p_k}&\\
\end{align*}

\begin{align*}
\varphi(n)
&=n\prod_{i=1}^{n}\left(1-\frac{1}{p_i}\right)&\\
&=n\prod_{\substack{p\divides n\\p\colon\text{primo}}}\left(1-\frac{1}{p}\right)&\\
\end{align*}
\end{proof}
\end{theorem}

\subsection{Algunas propiedades de la función indicador}

\begin{enumerate}
	\item $\varphi\left(p^{\alpha}\right)=p^{\alpha}-p^{\alpha-1}$
	\item $\varphi\left(mn\right)=\varphi(m)\varphi(n)\left(\frac{d}{\varphi(d)}\right)$
	\item $\varphi\left(mn\right)=\varphi(m)\varphi(n)$, si $\mcd(m,n)=1$.
	\item Si $a\divides b$, entonces $\varphi(a)\divides\varphi(b)$.
	\item $\varphi(n)$ es par para cada $n\geq3$. Más aún$\colon$
	\[n={p}^{\alpha_1}_{1}{p}^{\alpha_2}_{2}\cdots{p}^{\alpha_k}_{k}\implies 2^{k}\divides\varphi(n)\]
\end{enumerate}


\section{Funciones multiplicativas}

\begin{definition}
Una función aritmética $f\colon\mathbb{Z}^{+}\rightarrow\mathbb{C}$ es llamada multiplicativa si $f$ no es idénticamente nula y si

\[f(mn)=f(m)f(n)\quad\forall m,n\in\mathbb{Z}^{+},\quad\mcd(a,b)=1.\]

\end{definition}

\begin{definition}
Una función multiplicativa $f\colon\mathbb{Z}^{+}\rightarrow\mathbb{C}$ es completamente multiplicativa si:

\[f(mn)=f(m)f(n)\quad\forall m,n\in\mathbb{Z}^{+}.\]
\end{definition}

\begin{example}
$\varphi(n)$ es multiplicativa, pero no es completamente multiplicativa $\forall n\in\mathbb{Z}^{+}$.
\end{example}

\begin{theorem}
Si $f$ es multiplicativa, entonces $f(1)=1$.
\begin{proof}
\[\cancel{f(n)}=f(n\cdot1)\ddef \bcancel{f(n)}f(1)\implies f(1)=1.\]
\end{proof}
\end{theorem}

\begin{theorem}
Sea $f\colon\mathbb{Z}^{+}\rightarrow\mathbb{C}$ con $f(1)=1$.
\begin{enumerate}
	\item $f$ es multiplicativa $\iff f\left({p}^{\alpha_1}_{1}{p}^{\alpha_2}_{2}\cdots{p}^{\alpha_k}_{k}\right)=f\left({p}^{\alpha_1}_{1}\right)\cdot f\left({p}^{\alpha_2}_{2}\right)\cdot\cdots\cdot f\left({p}^{\alpha_k}_{k}\right)\quad\forall p_1p_2\cdots p_k$ primos y $\alpha_1,\alpha_2,\ldots,\alpha_k\leq1\in\mathbb{Z}^{+}$.
	\item Si $f$ es multiplicativa, entonces

	\[f \text{ es completamente multiplicativa }\iff f\left(p^{\alpha}\right)={\left(f(p)\right)}^{\alpha}\]
	$\forall p\colon$ primo y $\tilde{\alpha}\in\mathbb{Z}^{+}$.
\end{enumerate}
\end{theorem}