\chapter*{Un poco acerca de la historia de la \emph{Teoría de números}}

En el año 1912, durante el quinto 
\href{https://www.mathunion.org/organization/imu-history}{Congreso Internacional de Matemáticos}, el matemático alemán, en tal evento, \href{https://en.wikipedia.org/wiki/Edmund_Landau}{Edmund Landau} listó cuatro problemas básicos acerca de \emph{los números primos} que los calificó como \emph{inabarcable en el estado actual de las matemáticas} y en nuestros días es conocido como los \textbf{\textit{Problemas de Landau}}. Ellos son los siguientes:
\begin{enumerate}
	\item Conjetura de Goldbach\footnote{Helfgott probó la conjetura débil en el año 2013.}
	\item Conjetura de los primos gemelos.
	\item Conjetura de Legendre.
	\item ¿Existen infinitos números primos de la forma $n^{2}+1$?
\end{enumerate}

Ninguno de los cuatro problemas han sido resultados a la fecha de la edición de los apuntes.

La teoría de números tal como se conoce en la actualidad inició desde Gauss. En el año 2013, el matemático peruano Harald Andrés Helfgott demostró la \emph{Conjetura débil (o ternaria) de Goldbach}
%https://www.gaussianos.com/harald-andres-helfgott-nos-habla-sobre-su-demostracion-de-la-conjetura-debil-de-goldbach/
\section{Harald Andrés Helfgott}
\begin{conjecture}{Goldbach}
Todo número impar mayor que cinco es la suma de tres números primos.
\end{conjecture}
\emph{Leonhard Euler}, uno de los más grandes matemáticos del siglo XVIII, y de todas las épocas, y su amigo cercano \emph{Christian Goldbach}, quien tenía grandes conocimientos tanto en la ciencia como en las humanidades, mantuvo una regular y copiosa correspondencia. Goldbach hizo una conjetura acerca de los números primos, y Euler rápidamente redujo a la siguiente conjetura, el cual, dijo, Goldbach ya le había dicho: cada entero positivo puede escribirse a lo más, como la suma de tres números primos.

%En el año X, Terence Tao probó para el caso como la suma de cinco números primos.

Ahora, podríamos decir que ``cada entero mayor que cinco'', ya que no consideramos a 1 como un número primo. Es más, en la actualidad la conjetura se divide en dos casos: la \emph{conjetura débil o ternaria de Goldbach} establece que cada entero impar mayor que cinco se puede escribir como la suma de tres primos y la \emph{conjetura fuerte o binaria de Goldbach} establece que cada entero par mayor que 2 se puede escribir como la suma de dos primos. Como indican sus nombres, la conjetura fuerte implica la débil (fácilmente: reste 3 al número impar $n$, luego exprese $n-3$ como la suma de dos números primos).
%https://valuevar.wordpress.com/2013/07/02/the-ternary-goldbach-conjecture/
%http://libgen.io/search.php?&req=History+of+the+theory+of+numbers+Dickson&phrase=1&view=simple&column=def&sort=year&sortmode=DESC
%Cayley y Sylvester trabajaron en la conjetura de Goldbach.
%Citar el libro de Escalante sobre las figuras.
%Grandes matemáticos de Temple Bell.

\begin{figure}[H]
	\centering
	\def\svgwidth{7cm}
	\input{imo2018.pdf_tex}	
\end{figure}

